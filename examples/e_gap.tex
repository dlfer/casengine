\documentclass[a4paper,twoside,leqno]{article}
\usepackage{amsmath,amsthm,amssymb,amsfonts}
\usepackage{mathpazo} % I like it.
\usepackage[CAS=gap]{casengine}

\newcounter{exer}
\numberwithin{exer}{subsection}
\renewcommand{\theexer}{(E\arabic{exer})}
\renewcommand{\thesubsection}{\Alph{subsection})}
\newtheoremstyle{exer}% name
     {24pt}%      Space above
     {24pt}%      Space below
     {}%         Body font: it was: \itshape
     {}%         Indent amount (empty = no indent, \parindent = para indent)
     {\bfseries}% Thm head font
     {}%        Punctuation after thm head
     {.5em}%     Space after thm head: " " = normal interword space;
               % \newline = linebreak
     {}%         Thm head spec (can be left empty, meaning `normal')

\theoremstyle{exer}
\newtheorem{exe}[exer]{}


\begin{document}

\section{Exercises}

\symexec{elmG:=Elements(SymmetricGroup(4));}
\begin{symfor}{q}{range(1,3)}%
\begin{symfor}{c}{range(1,3)}%
\begin{symfor}{x}{(1,2);(2,3);(3,4);(3,4,2);(1,2,3,4);(1,2,3);(3,4,1)}%

\symexec{g:=elmG[q+1]*x^elmG[c+1];}
\begin{exe}
Try to compute 
\[
\sym{elmG[q+1]} \sym{x}^{\sym{elmG[c+1]} } = ...
\]
Answer: $\sym{g}$. 
\end{exe}
\end{symfor}

\end{symfor}
\end{symfor}


\end{document}
